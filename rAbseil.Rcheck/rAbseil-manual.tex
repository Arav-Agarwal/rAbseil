\nonstopmode{}
\documentclass[letterpaper]{book}
\usepackage[times,inconsolata,hyper]{Rd}
\usepackage{makeidx}
\usepackage[utf8,latin1]{inputenc}
% \usepackage{graphicx} % @USE GRAPHICX@
\makeindex{}
\begin{document}
\chapter*{}
\begin{center}
{\textbf{\huge Package `rAbseil'}}
\par\bigskip{\large \today}
\end{center}
\begin{description}
\raggedright{}
\item[Type]\AsIs{Package}
\item[Title]\AsIs{What the Package Does in One 'Title Case' Line}
\item[Version]\AsIs{1.0}
\item[Date]\AsIs{2020-06-22}
\item[Author]\AsIs{Your Name}
\item[Maintainer]\AsIs{Your Name }\email{your@email.com}\AsIs{}
\item[Description]\AsIs{One paragraph description of what the package does as one
or more full sentences.}
\item[License]\AsIs{GPL (>= 2)}
\item[Imports]\AsIs{Rcpp (>= 1.0.4)}
\item[LinkingTo]\AsIs{Rcpp}
\end{description}
\Rdcontents{\R{} topics documented:}
\inputencoding{utf8}
\HeaderA{rAbseil-package}{A short title line describing what the package does}{rAbseil.Rdash.package}
\aliasA{rAbseil}{rAbseil-package}{rAbseil}
\keyword{package}{rAbseil-package}
%
\begin{Description}\relax
A more detailed description of what the package does. A length
of about one to five lines is recommended.
\end{Description}
%
\begin{Details}\relax
This section should provide a more detailed overview of how to use the
package, including the most important functions.
\end{Details}
%
\begin{Author}\relax
Your Name, email optional.

Maintainer: Your Name <your@email.com>
\end{Author}
%
\begin{References}\relax
This optional section can contain literature or other references for
background information.
\end{References}
%
\begin{SeeAlso}\relax
Optional links to other man pages
\end{SeeAlso}
%
\begin{Examples}
\begin{ExampleCode}
  ## Not run: 
     ## Optional simple examples of the most important functions
     ## These can be in \dontrun{} and \donttest{} blocks.   
  
## End(Not run)
\end{ExampleCode}
\end{Examples}
\inputencoding{utf8}
\HeaderA{rcpp\_hello\_world}{Simple function using Rcpp}{rcpp.Rul.hello.Rul.world}
%
\begin{Description}\relax
Simple function using Rcpp
\end{Description}
%
\begin{Usage}
\begin{verbatim}
rcpp_hello_world()	
\end{verbatim}
\end{Usage}
%
\begin{Examples}
\begin{ExampleCode}
## Not run: 
rcpp_hello_world()

## End(Not run)
\end{ExampleCode}
\end{Examples}
\printindex{}
\end{document}
